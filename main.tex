% This document is copyright 2025, 2026 the author. All rights reserved.

% to-do and questions for Hogg
% ----------------------------
% - finish writing the stuff
% - are there recommendations re LLMs?
% - postmodern idea that the meanings of documents are set by the *readers* not the *writers*. How does this connect?
% - NBH point that we don't know the *provenance* of words written by the LLM; hence they violate our principles of provenance. Cite her blog post.
% - On code, and use for coding, cite Bloom blog post?
% - look at paragraph breaks?

\documentclass{article}

% typesetting and text macros
\usepackage[letterpaper, textheight=9.0in, textwidth=5.2in]{geometry}
\usepackage{setspace}
\setstretch{1.08}
\sloppy\sloppypar\raggedbottom\frenchspacing
\newcommand{\documentname}{\textsl{white paper}}
\newcommand{\sectionname}{Section}
\newcommand{\secref}[1]{\sectionname~\ref{#1}}
\newcommand{\notename}{note}
\newcommand{\noteref}[1]{\notename\textsuperscript{\ref{#1}}}
\newcommand{\paragraphskip}{\medskip}
\renewcommand{\paragraph}[1]{\par\paragraphskip\noindent\textbf{#1}}
\bibliographystyle{unsrt}\renewcommand{\newblock}{}
\let\OLDthebibliography\thebibliography
\renewcommand\thebibliography[1]{
  \OLDthebibliography{#1}\setlength{\parskip}{0pt}\setlength{\itemsep}{3pt plus 0.3ex}\footnotesize\raggedright%
}

\title{\bfseries Why do we do astrophysics?}
\author{\textbf{David W. Hogg}\footnote{%
David W. Hogg is in the Center for Cosmology and Particle Physics, Department of Physics, New York University; and in the Center for Computational Astronomy, Flatiron Institute; and at the Max-Planck-Institut f\"ur Astronomie, Heidelberg.}~\footnote{%
The author thanks
  Mike Blanton (Carnegie),
  Gaby Contardo (Nova Gorica),
  Matt Daunt (NYU),
  Juna Kollmeier (Carnegie),
  Jim Peebles (Princeton),
  Adrian Price-Whelan (Flatiron),
  Hans-Walter Rix (MPIA), and
  Kate Storey-Fisher (Stanford)
for discussions that informed the ideas presented here, and 
  Natalie B. Hogg (Cambridge),
  Phil Marshall (SLAC), and
  Kate Storey-Fisher (Stanford)
for discussion of the particular content of this \documentname.
None of these people are responsible for any of the claims presented here.
The Flatiron Institute is a division of the Simons Foundation.}}
\date{January 2026}

\begin{document}

\maketitle

\paragraph{Abstract:}
At time of writing, large language models (LLMs) are beginning to obtain the ability to design, execute, write up, and referee scientific projects on the data-science side of astrophysics.
What implications does this have for our profession?
In this \documentname{},
I list---and argue for---a set of facts or ``points of agreement'' about what astrophysics is, or should be.
I discuss a set of putative benefits that astrophysics perhaps brings to us, and to science, and to universities, and to the world.
I conclude with possible policy recommendations, none of which I endorse.

\section{Introduction}\label{sec:intro}
This document is being written at a moment of very rapid change.
What are currently called ``large language models'' (LLMs)---%
or sometimes ``generative artificial intelligence'' (although I really don't like that)---%
can now design, plan, execute, write up, and referee astrophysics projects.
At time of writing, one scientific-paper-by-LLM project has already been published \cite{denario} and I expect more in the category to follow.
There are already conferences being set up in which all the papers are written, and all the refereeing is performed, by machines \cite{conference}.
This \documentname{} is inspired by this very rapid change.

There is another change happening in astrophysics, however, which is happening on a much longer time scale, but is relevant:
Astronomical data production is becoming extremely professionalized, and in a very particular way.
An extreme example is the ESA \textsl{Gaia} Mission \cite{gaia}:
Once the instrument was designed, the plans were moved inside the fence\footnote{I am using the phrase ``inside the fence'' to mean ``accessible only to those with governmental approval to work on weapons systems, that is, those with security clearances.''} at EADS, the European military contractor.
EADS then figured out how to build it, built it, launched it, and operated it.
No research-active astronomers were involved in any part of that operation; indeed even the scientific members of the \textsl{Gaia} Data Processing and Analysis Consortium (DPAC; \cite{dpac}) are in the position of having to reverse engineer the algorithms that made the on-board decisions (on the \textsl{Gaia} spacecraft) about what to window and telemeter to ground.
Don't get me wrong: \textsl{Gaia} has been incredibly important to global astrophysics (and my own research program); it is probably the most productive astrophysics space mission in science-per-dollar terms in the last decade; but it wasn't built or operated by \emph{astronomers}.\footnote{%
I will use the words ``astronomer'' and ``astrophysicist'' almost interchangeably here. For me the distinction is soft, and related to \emph{intentions}:
Astronomers want to measure things in the sky precisely. Astrophysicists want to derive physical insights from those measurements.
Most of us do a bit of both (I hope).
Both astronomers and astrophysicists contribute to and criticize the scientific literature. One could object to my claim that \textsl{Gaia} was not built by astronomers by saying that any employee of EADS who worked on \textsl{Gaia} is obviously an astronomer:
They built a telescope!
But they are not astronomers in the sense of contributing to and criticizing the astronomical literature.
Indeed, most employees at EADS are not permitted to discuss their work in public, because their work relates to weapons systems.}
Astronomers, in the case of \textsl{Gaia}, are just end users; end users of curated, calibrated data, delivered by a combination of the (secret) spacecraft and the (absolutely great, professional, and open) DPAC.

The \textsl{Gaia} Mission is an extreme example, but many large projects are in this category, in one way or another:
Most of the NASA \textsl{JWST} \cite{jwst} instruments were built and delivered by defense contractors.
The enormous \textsl{LSST} project \cite{lsst} is being operated by a team that is a mixture of astrophysicists and engineering professionals.
Even university-based projects like the \textsl{Sloan Digital Sky Survey} projects \cite{sdssiv, sdssv} or \textsl{GALAH} \cite{galah} endeavor to produce science-ready data products that can be queried through application programming interfaces, plotted, and analyzed without much worry about where they came from or how they got here.
There are two points in play here:
The first is that teams that \emph{produce data} are increasingly professionalized, and removed from research astrophysics.
The second is that the astrophysics community is increasingly expecting to receive trustworthy, vetted, calibrated, and complete data.

I have been involved in bringing about this change (for example, \cite{monitor, ubercal, astrometrynet, opendata}), and in many ways it is absolutely great.
It democratizes astronomy, since it lowers barriers to entry.
It creates an open-science space, in which every project benefits from the output of every other project.
It makes it very easy to make discoveries, and especially discoveries that involve multiple data modalities.
It thus weakened the walls of the astronomical silos (``radio astronomer,'' ``spectroscopist,'' etc), and especially blurred the distinctions between the categories of ``theorist'' and ``observer.''
But it does have a strange consequence, which is also related to professionalization:
For some kinds of projects in astrophysics, there isn't a huge difference in capability between a classically trained astronomer and a newly trained \emph{data scientist}.
Indeed, in many ways, the data scientist is better prepared to work in astronomy than the astronomer, since the data scientist knows how to do fast computation on large data sets, build complex probabilistic models, and perform large-scale optimizations or searches.
For analysis of the \textsl{Gaia} data, a data scientist who has taken an astronomy class might be better prepared than an astronomer who has taken a data science class.

Of course much of astrophysics is \emph{not} like working with the \textsl{Gaia} data.
Many astrophysicists build instruments, operate observatories, and theorize.
Even a student observing at a small, university-based observatory is doing things that go way beyond data science, in assessing conditions, and deciding things about data quality, or making on-the-fly changes to the observing schedule based on weather or issues at the telescope.
Not all of astrophysics is data science, not by a long shot!
But a lot of it is, and the fraction is increasing every year.

How are these two changes---LLM successes and the data-scientification---related?
They are related because \emph{contemporary LLMs can do data science}.
It's true that there are many known limitations of present-day LLMs, for example in doing arithmetic \cite{arithmetic}, or in performing causal inferences \cite{causal}, but the models are improving by the week, and I think it is reasonable to expect that they will steadily become more competent at designing, executing, writing up, and criticizing scientific projects, at least for a while.
There might be show-stoppers; we don't know yet.
But this \documentname{} is being written under the assumption that LLM-like models (or ``genAI'') will improve with time until they can perform data science tasks with the same proficiency as humans.
What does the study of astrophysics look like in this future?
I don't know; this \documentname{} isn't even going to answer that.
All I am going to ask is: Why do we do astrophysics?
I have the strong feeling that we will need to answer this question first, before we can answer questions about our relationships with LLMs and LLM-based projects.

\section{Points of agreement}\label{sec:facts}
We'll start here by stating some facts about astrophysics, or points on which we can all agree.
Of course for each of these, there are many astrophysicists who will disagree.
But these form a set of assumptions from which I will try to flow any recommendations I might make.

\paragraph{Astrophysics produces new knowledge about the Universe.}
Astrophysics is a science that is driven by discovery and new measurement.
Every PhD project and (almost) every paper involves measuring something that hasn't been measured before, or in a different way, or making a new interpretation or prediction, or improving a methodology, or finding a new object or set of objects.
That is, every project that counts as ``astrophysics'' involves scientific novelty.

If you want to become an astrophysicist, it isn't sufficient to read about it, or take classes in it.
You have to \emph{do it}, and \emph{doing it} requires doing novel things, that haven't been done before, and which connect to important scientific questions in the literature.

It isn't even sufficient to repeat classic or known measurements.
Part of becoming an astrophysicist is learning how to design and plan new kinds of measurements or new approaches or applying known techniques in new areas to make new discoveries.
The novelty is explicitly part of the project; astrophysics isn't just techniques or results; it intrinsically involves innovation.
Astrophysics lives at the cutting edge,\footnote{%
I owe a lot to Rosie Wyse (JHU), in general, but also in particular for making this point very clearly to me in a conversation in 2023.}
not in the past.

\paragraph{People are always the ends, not merely the means.}
It might sound trite to say that we do astrophysics for people.
But the consequences of this are at least a little bit non-trivial:
When we employ a graduate student to perform some work, it absolutely must be \emph{because the graduate student will benefit} from that work, not merely because that work needs to get done.
This is one of the interpretations of the categorical imperative \cite{kant}, which is (to my mind) one of the few undeniably true principles or laws that have ever been written in the philosophical subject of ethics.

I have heard it said, more than once, that an LLM can do some task ``better than a graduate student.''\footnote{%
One memorable occasion was Paco Villaescusa (Flatiron) saying something even substantially stronger than this, publicly, in a room full of graduate students, during a seminar in the Department of Physics at New York University in 2025.}
That language makes me uncomfortable, because it is taking an extremely \emph{instrumental} view of graduate students.
Are graduate students in our business \emph{to do work}?
Or are they here \emph{to learn}?
I very much hope the latter, or, if they are here to work, it is because that work is also critical to their learning.
We train PhD students not merely to amplify our own research programs, but to create opportunities, and specifically opportunities for them.
In general, it is possible to take a very instrumental view of the world, in which people are described in terms of their functions---their input--output relationships, if you will---and this is the view of the world I often see reflected in the press releases of robotics and artificial-intelligence companies.
It is important that we reject this view.
Every person is a human being, whose personal development is more important than our short-term scientific accomplishments.\footnote{%
One could make an argument that our \emph{long-term} scientific accomplishments---the theory of gravity, the discovery of dark matter, the possible future understanding of the origin of life or the discovery of life outside the Solar System----might be as important as people, in some sense.
But individual things that PhD students are doing are way less important than those PhD students themselves.}

You might object: But science is \emph{also} important!
It is, perhaps, but see below my point about clinical value.
If a science does not have clinical value (no way to help people), then its short-term results are definitely and absolutely less important than its people.
If a science \emph{does} have clinical value, there might be room for debate.
You might notice that I am not a \emph{utilitarian}, but I am willing to listen to utilitarian arguments.\footnote{%
Utilitarian arguments balance benefits and harms to people, with the goal of finding the solutions that most increase the ``sum total of human happiness'' \cite{mill}.
If a science has clinical value, then the harm to workers (of being forced to do things they don't want to do, say, or being underpaid for their work, say), can be balanced by the good done to the recipients of the clinical successes.
I don't like these arguments, because they are used to justify using people as a means, and not an end.
Utilitarianism works well with capitalism and development.}

One interesting possible consequence of this point---the point that people are the ends, not the means---is that the standard form for research proposals might be unethical:
Research funding in those proposals pays for graduate students, with the explicit promise of deliverables from those students.
When it's crunch time, are we obeying the categorical imperative?
Another interesting possible consequence is that \emph{scientific embargoes might be unethical}.
After all, how does stopping early-career scientists from talking publicly about their work serve \emph{their} interests?

\paragraph{Astrophysics is (roughly) the astrophysics literature.}
My group is, in some sense, a software group.
We produce software systems that many people use (for example, \cite{astrometrynet, emcee}).
For this reason, I am occasionally asked to speak on the idea that software is just as important as other kinds of scientific outputs.
People are surprised when I don't completely agree.
I believe that (in astrophysics) software is written to support the astrophysics literature, and that every important piece of software should have an associated paper in the astrophysics literature.
That paper describes the intellectual contributions encoded in the software, and provides a standardized mechanism for tracing intellectual provenance, giving credit, and coordinating criticism.
Maybe my position is radical now?
But I believe that software is important in that it embodies and executes \emph{our ideas and beliefs} about astrophysics.
These ideas and beliefs are well gathered, disseminated, and preserved in the astrophysics literature.\footnote{%
I should say here that I take the dissemination and preservation aspects of the literature more seriously than any other aspects.
For this reason, I believe that \textsl{arXiv} is every bit as important as \textit{The Astrophysical Journal}.
However, I believe that low-cost ``free journals'' that just provide a ``refereeing layer'' but don't actually preserve the literature, and don't actually publish content, don't contribute much, since they are neither a disseminator nor a preserver of literature.
Preserving and disseminating knowledge is never inexpensive.}

Nothing we have has the track record that the traditional journals do for long-term dissemination and preservation of knowledge; we don't know what computing hardware platforms we will have in 15 years, let alone 200, but I can pretty-much guarantee that---if human civilization is still in existence---we will still be able to read Vera Rubin's papers about the dark matter (for example, \cite{vera}).\footnote{%
One super-weird thing that is definitely off-topic here is that the journals no longer have print (hard-copy) form.
It used to be (when I was a PhD student) that every published scientific paper had a printed copy, on acid-free paper, in almost every academic library in the world (literally thousands of copies, globally distributed, and well tended).
Now there are as many copies of an astronomical paper as the main journals have backups.
How many is that? Three?
We are one small-scale war away from losing everything that astronomers have done this entire century.
That's why my example paper is Vera Rubin's, not any of my students'.}

But that's all just introduction to the main point here, which is that astrophysics \emph{is} the astrophysics literature.
If you want to ask ``what is known about the structure of neutron stars?'' you ask the literature (scientific articles plus sometimes textbook chapters).
Even if you ask a person, that person justifies their answer by pointing to the literature.
If an astronomer measures something, but never publishes that measurement, then that measurement does not get disseminated, known, and preserved.
It does not become part of astrophysics.
That is, the literature is the repository of all of our knowledge and the only reliable authority.

I will make two small comments about this.
The first is that astrophysics, like any science, contains a lot of ``implicit knowledge'' or folklore about things like how to observe, how to reduce data, how to organize projects, how to visualize data and models, how to read and write, and so on.
Much of this never appears in the literature.
Is that not also astrophysics?
Yes it is, but it is \emph{astrophysics practice}.\footnote{I would welcome a project in which we tried to make much of this implicit knowledge explicit.}
The results of astrophysics---the scientific conclusions and debates---are in the literature.

The second comment is that I often hear software (and hardware and engineering-oriented) people say that they ``have to'' write papers because papers---and the citations that they generate---are ``the coin of the realm'' (for example, \cite{coinofrealm}).
Papers (and the authorships on those papers) and the citations of those papers are not ``coin'' of anything!
They represent our recording of what happened, what we know, and how we know it.
Citations deliver provenance, not reward.\footnote{%
The authors on a paper should be those responsible for creating the results described in that paper.
Papers should cite the papers that are relevant to their results---the papers they are criticizing or building on.
Anything else unethically distorts the provenance of the ideas and work in our field.}
The literature records what we know and how we know it.

\paragraph{We must use our resources efficiently.}
Partly because we deliver no clinical value (see below), essentially all astrophysics grants and funding are \emph{gifts}.
Our funding comes from a combination of public purses (national agencies, public universities, national and international observatories and spacecraft) and private sources (foundations, private universities, individual donors).
I don't understand the motivations of most of these actors, if they can even be described as having motivations at all.
However, if we squander these funds, the sources will dry up.
Perhaps more importantly, when the source of our funding or support is public, we owe it to the people represented by that public entity to use their contributions wisely and well.
Even with private funding, the source is, ultimately, the people.

It is an absolute requirement that we not waste money.
Therefore, if there are a few different ways to make the same measurement, it is incumbent on us to choose the most efficient of them, unless there is some important intellectual value in making the measurement in multiple ways (as there sometimes is).
We should work with the wavelengths, the techniques, the people, the technologies, and instruments that are best at making our measurements.

This point is connected to the ``learn new things'' point above, but it is worth noting that it is slightly in conflict with the ``people are the ends'' point.
After all, graduate students aren't \emph{labor} to be \emph{efficiently deployed}.\footnote{%
The first person to clearly say this to me was Mike Blanton (Carnegie), during a conversation at NYU about the efficiency of the \textit{Sloan Digital Sky Survey IV}.}
Graduate students are human beings, for whom we are supposed to be creating new opportunities.
Importantly for what follows, however, if LLMs can do something important, efficiently, it is not obvious that we can just ignore that.

\paragraph{Correctness and rigor are paramount.}
Astrophysics is an observational science.
We like to talk about ``experiments,'' but fundamentally we can't perform controlled experiments:
We observe the Universe as it is, with the photons that happen to enter the apertures of our telescopes during the times our shutters happen to be open.
Thus exact reproducibility is difficult, and some observations cannot be verified, no matter what we do.\footnote{%
For example, the very smallest, longest-period planets we found with the NASA \textsl{Kepler} Mission \cite{keplerresults} represent very low-amplitude, very rare photometric events.
They are essentially impossible to confirm with any independent data, short of launching another \textsl{Kepler}-like mission.
For another example, the NASA \textsl{WMAP} \cite{wmap} and ESA \textsl{Planck} \cite{planck} Missions measured everything that can be measured in the primary, scalar fluctuations in the cosmic microwave background.
How can we tell whether these results are correct?
Either we inspect their pipelines, or else we launch another billion-USD mission.
If those points don't convince you, consider supernovae:
No supernova ever repeats, so it would literally require a violation of Lorentz invariance to verify any observations, light echos \cite{lightecho} notwithstanding.}
Thus we check or ``confirm'' most measurements in astrophysics just \emph{by inspection}, inspection of the data and code that were taken and used in the generating the analysis and conclusions.

The only authority we have for what we do lies in methodological domains adjacent to astronomy:
We rely on optics, mechanical engineering, remote sensing, statistics, and---very importantly---applied mathematics.
If we don't use these methods correctly, we will get wrong answers, and, in the daisy chain of reasoning that is astrophysics, those wrong answers will propagate through to other, downstream results.
It is imperative that we do our work as correctly as possible, and that we understand and admit the approximations and mistakes that we are making along the way.
We must test those assumptions and correct those mistakes, perhaps not immediately, but over time, and as a community.
We must call out mistakes and criticize wrong results and bad assumptions, and we must re-do analyses that we believe to be flawed.
If we stop doing things rigorously and correctly, we have stopped doing astrophysics.
This is especially true in a science with \emph{no clinical value}.

\paragraph{Astrophysics has no clinical value.}
I like to say that the sciences have a ``left edge'' which is about fundamental understanding, and understanding for understanding's sake.
They also mostly have a ``right edge'' which is about clinical value or application or use in the world.
So, for example, biology has a left edge on which people understand how the cytoskeleton moves the cell, and a right edge in which they develop or improve therapies for cancer or cloudy toenails.
Macroeconomics has a left edge that involves relationships between unemployment and inflation, and a right edge that informs government policy about interest rates.
I claim (and maybe this is a bit controversial) that \emph{astronomy has no right edge}.
That is, there are no useful things in the world that flow from astronomical discoveries and results.
I have spent years of my life estimating the comoving volume of the Universe \cite{phdthesis}, measuring the local dark-matter density \cite{oti}, and finding planets around other stars \cite{singletransit}.
No human outcome or pragmatic capability has been affected in the slightest by any of my results.
Literally nothing helpful to humanity arises here.

I speak about this sometimes, and I hear objections of various kinds.
One silly objection is that we might find ``Planet B.''
Nope. We're all dying---and every human who is ever born will be dying---on this rock, or extremely close to it.\footnote{There are many reasons this is true; some of them are discussed in \cite{nospacetravel}. It is also very eloquently stated by Michel Mayor in his acceptance speech for the 2019 Nobel Prize in Physics.}
A less silly objection is that the results of astrophysics inform particle physics.
They do!
We learned about neutrino mass mixing and the entire dark sector from astrophysics, not particle experiments.
But these astrophysics results inform the \emph{left edge} of particle physics, not the right edge.
An even less silly objection is that we develop, build, and exercise imagers, cameras, spectrographs, spacecraft platforms and so on, which have right-edge applications.
All that is true!
That's the closest we come to having a right edge:
We contribute to precision remote-sensing technologies, which, in turn, help the US and other nations target weapons.
Congratulations, astronomy!
But I don't even count this as clinical use of astronomy, because we build these tools \emph{to do astronomy}, these aren't the \emph{results of astronomy}.
My point about astronomy having no right edge is that nothing in the world of things or people hangs on the precise value of the age of the Universe.
Many astronomers have spent many years (and billions of dollars) measuring it, and we know it very precisely \cite{age}.
But literally nothing hinges on the question of whether it is 13.77 billion years or 13.79.
So feel free to disagree here, and say ``yes, astronomy does have a right edge; we improve the targeting of weapons.''
I won't fight you.
But, for the purposes of my arguments, this won't be the kind of right edge that affects how we do things.

One way to see this is that even if our detectors and satellites were useless to any other industry,
we wouldn't use them differently, or do anything differently.
Note the difference with biology: If we found that cancer could not ever be cured with biological resources, a large part of the study of biology would change overnight.
After all, much of biology is justified in these right-edge terms.
None of astrophysics is justified in these right-edge terms.
No astronomer (that I know) is improving the calibration of \textsl{JWST} instruments because they want the US Navy to have a higher kill rate.\footnote{%
The closest thing I can see in my own world to anyone making this argument is that the US funding agencies (NASA and NSF) explicitly prefer research proposals that develop technologies that could be transferred to commercial enterprises.}

To this no-clinical-value point I have heard the objection:
Since (as we said above) humanity loves astrophysics, it must be the case that it is clinically valuable!
It must be that the clinical value lies in its feeding of humanity's love.\footnote{%
Roger Blandford (Stanford) sometimes says that astrophysics provides ``hope'' to humanity \cite{hope}.
Maybe that's a strong claim, but I don't think it is wrong.}
But this is not truly a \emph{clinical} value in the following sense:
In no way does this application of astrophysics depend on the correctness or values of the outcomes of astrophysics projects.
For example, if biology learns something new about a protein interaction, that directly affects a drug-development path.
If that result turns out to be wrong, that drug-development path will fail.
The clinical side of biology \emph{tests} biological results.
Similarly, if the macroeconomics community gets the relationship between unemployment and inflation wrong, policy adjustments will fail to work.
A science has a right edge if and only if the associated clinical work actually tests or exercises the specific results of the science.
In this sense, astrophysics has no right edge.

\section{What's good about astrophysics?}\label{sec:benefits}
\secref{sec:facts} was supposed to list some facts, or points of agreement, about what astrophysics is,
or what it should be.
One of those facts is that astrophysics has no clincal value---its results provide no useful resources to humanity.
That doesn't mean there aren't extremely important benefits of doing astrophysics.
This \sectionname{} is about those benefits.
What do we deliver to the world?
I am going to list all the things I know here, but (as you will see) I don't agree with all of them.

\paragraph{Everyone loves astronomy.}
When I decided to work in astrophysics (which I did in the period 1990--1993), I remember my father saying that he felt like the area is a good one, because astronomy will always be studied by humanity, no matter what.
The interest in astronomy is so deep, relevant research has been done by countless civilizations, as long as humans have been around (and probably before).
In the same way that it is good that humans produce (and read) novels, and poetry, it is good that humans produce (and learn) new ideas in astronomy and astrophysics.

\paragraph{Physics needs astrophysics.}
As mentioned above, there are many things we can't understand about physics without doing astrophysics.
The highest vacuums, strongest electromagnetic fields, and largest length scales are only accessible astronomically.
The theory of gravity was found and is entirely informed (except perhaps \cite{adelberger}) through astrophysics, including black holes, gravitational radiation, and accretion flows.
The dark matter and vacuum energy density---and whole dark sector, whatever it is---is detectable (so far) only through stellar and cosmological dynamics.
Those are all aspects of fundamental physics that are available through astrophysics, and only astrophysics.

There are other physics questions that are less fundamental (perhaps), related to our \emph{origins}:
How did life form? How did Earth form? How did the Sun form?
These are critical physics questions, and they also depend crucially on astronomical observations and astrophysical interpretation.

\paragraph{Universities need astrophysicists.}
We put ``the universe'' in ``the university''?
A university is a place where students can study and learn \emph{anything} of importance to humanity.
Astrophysics is important to humanity.
Therefore it is essential that universities have astrophysicists in faculty positions.
Inasmuch as the research university is a good idea, astrophysicists and the study of astrophysics is a good idea.

\paragraph{We train a technical workforce.}
Very few of my PhD students---and very few of anyone's PhD students, I hope---become research astrophysicists.
Most astrophysics PhDs obtain long-term careers in other technical professions, such as tech, finance, and engineering.
Astrophysics, it turns out, is a very good training for quantitative professions that involve computational modeling.
Thus astrophysics research produces very good people for technical innovation and growing our economy.
This technical workforce argument is literally and explicitly the reason that the Max Planck Society in Germany supports astrophysics research \cite{mpi}.

Although PhD students are the group most ``trained in astrophysics,'' all the same arguments apply to undergraduates.
Undergraduate astrophysics students go on to technical careers in all kinds of industries,
and also populate graduate schools in many natural-science and engineering disciplines.
An astrophysics undergraduate degree is a great preparation for work in climate, for example, which is undeniably important.

\paragraph{We educate the public.}
The public---from undergraduates \emph{not} in the sciences to the television-watching public---are excited about astrophysics and want to learn more about it and more from it.
I don't think it is an exaggeration to say that most members of society are interested in the beginning of the Universe, the properties of black holes, and the origin of life.
Heck, we might even discover aliens some day.
If the interests of the public matter, then we are doing something important to many, and contributing rich content to public education.
One thing many US astrophysicists like to mention is that NASA regularly gets coverage on the front pages of the newspapers of record (or I guess homepages, nowadays).
Every one of those instances is a huge success of research astrophysics, and contributes to public understanding of (and excitement about) science.

\paragraph{We beat ploughshares into swords.}
The budgets of western nations\footnote{%
I don't want to exclude India, China, and Japan from this discussion, because I believe that the same things happen in Asia as well.
I just can't speak about it as confidently as I can about Europe and North America.}
(and especially the US) involve enormous contributions to military contractors, who in turn deliver extremely sophisticated technical capabilities for western-controlled weapons.
Despite this, there is a big and well-funded contractor lobby plus many lawmakers who believe that \emph{not enough} money is being delivered to these contractors.
Astrophysics projects create opportunities to use civilian budget (scientific research budget) on military applications.
If you want examples, look at the NASA and ESA roadmaps, which will show flying and near-future spacecraft, essentially every one of which delivers between hundreds of millions and billions of USD to weapons contractors.
This argument may make the reader uncomfortable---I sure don't like it---but this is absolutely one of the ``benefits'' of astrophysics, when one takes the view of ``national priorities.''\footnote{%
This ploughshares-to-swords activity is usually in play when boosters of NASA missions argue that, if we don't build and launch some particular mission, we will ``lose our national leadership'' in some technical area.
Those arguments are bad, I think, because they are primarily military arguments, and, also, they go against the upcoming ``borderless world'' argument.}

\paragraph{We use and develop remote-sensing capabilities.}
The astrometric and photometric precision requirements of contemporary astrophysics projects exceed those of remote-sensing projects in almost any other domain.
For this reason, astrophysics missions have been invaluable in the development of extremely stable (in an angular and thermal sense) spacecraft platforms, and of empirical models of the response functions of infrared detectors, for two examples of remote-sensing technologies.
We also exercise and contribute to the deep-space network, on-board compression and data management, and other engineering capabilities that assist with making complex measurements in low-bandwidth environments.
Many of the capabilities developed here are military, but not all are.
For example, ocean-observing missions \cite{oceanobserving} and precision tests of gravity \cite{gravityprobeb} have made use of technologies that were at least partially developed in astrophysics contexts.

\paragraph{We create opportunities for development.}
Despite the large budgets for some astronomical projects, we don't often think of ourselves as ``land developers.''
However, large projects on mountains in the US (including Hawai`i), China, and Chile (for examples), all represented development opportunities that contributed to economies (and often upset local populations).
In building they employ contractors and in operations they employ long-term staff, and make use of local utilities and services.
In some cases they improve local services or bring services to under-served locations.
In addition to these tangible effects, astronomical projects sometimes also effectively become monuments, representing national or regional pride (on the one hand, and colonial or state power on the other).

\paragraph{We create human knowledge.}
One of the tremendous accomplishments of humanity is its scientific knowledge, represented by the scientific literature, and also many other non-western records of science.
It is obviously a good thing that humanity has created, preserved, and transmitted this knowledge across space and time.
Thus our contributions to the astrophysics literature are obviously good:
They are contributions to a very good thing.

\paragraph{Astrophysics represents a borderless world.}
I have spent a large fraction of my career working on different generations of the \textsl{Sloan Digital Sky Survey} \cite{sdss, sdssiii, sdssiv, sdssv}.
These projects have become large, international collaborations with partners all over the globe.
In the recent generations, institutions were invited to become partners in these projects without regard to national or geographic location, and indeed there are partners on multiple continents.
Similarly, when we consider applicants for PhD programs, postdoctoral positions, and faculty jobs, we are open to applications from all over the world (although negatively constrained by visa or immigration requirements).\footnote{%
This is in stark contrast, by the way, to the scientific practice when I was an undergraduate.
The cold war and the iron curtain split the world into (at least) two disjoint pieces, even scientifically.
I met very few of my peers in China or Eastern Europe prior to 1992-ish.}

It isn't just about institutions and people, it is also the literature and ideas:
The journals accept paper submissions without regard to country of origin, and have the same refereeing standards for contributions from any place.
There are obviously national-origin biases among referees, and the literature is in English, and there complexities around page charges, but there are no explicit rules about country of origin for our principal journals.
Even more importantly, we don't (or certainly aren't supposed to) judge the correctness of a scientific claim based on the national origin of the scientist making the claim.

In all of these ways, astrophysics represents a borderless world, in which we demonstrate that borders are not required for the accomplishment of great things.
Just the opposite: The raising of walls is detrimental to science, as it impairs the network of collaboration, conferences, and the literature, which in turn compromises the mechanisms by which scientific ideas are tested, criticized, and elaborated.
Borders hinder, not help, the most important human activities and achievements.

\paragraph{Astrophysics is a satisfying activity.}
Finally, of course, \emph{astrophysics is fun}.
Most of us get involved because we are curious and like to solve problems and to build and do things.
The discovery of an object, the making of a measurement, the execution of a pipeline, the submission of a manuscript are all exciting and rewarding in themselves.
It is exciting to be a part of the community that figures out how the Universe works.
2019 Nobel Laureate Jim Peebles often speaks of the great success of ``curiosity-driven research'' in creating the empirical and theoretical bases for the physical model of cosmology that is ascendent today \cite{curiosity}.
I agree!

\paragraph{PEOPLE: What benefits have I missed?}

\section{Policy non-recommendations}\label{sec:policy}
I opened with the challenge confronting astrophysics with the arrival of increasingly competent large language models.
As I said, I won't end up actually making any policy recommendations here; this \documentname{} is a gathering of relevant ideas.
I will, however, suggest two possible, extreme policies; and I will argue \emph{against} both of them.
That is, I only have negative things to say about simple policies.
Before I get into that, I want to highlight one thing.

\paragraph{One (counterintuitive) reason we \emph{don't} do astrophysics:}
This might sound like a hot take, but I claim that (essentially) none of us, individually, does astrophysics because we want to learn the specific answer to the astrophysics question we are asking.
\emph{What?}
For example, when we measure the age of the Universe, no-one on the team actually cares what the specific value is, even though each of them might have spent many hard years of their life figuring out how to make the measurement correctly (and it isn't easy).

This point---that we don't care about the specific results---is obvious, in the following way:
Anyone who has the capability of getting a PhD in astrophysics has the capability of doing many remunerative things, substantially more remunerative than my job.
Thus, anyone who is doing astrophysics (professionally) could be, instead, earning enough money to pay \emph{multiple} professional astrophysicists to work on their behalf.
That employed astrophysics team would get to the answers of any astrophysics questions faster than the individual could by working on their own.
If all we \emph{really} wanted was to know how the Universe worked, we would start a hedge fund, and use the proceeds to pay an astrophysics institute, filled with people who wanted to \emph{do astrophysics} rather than \emph{find out the answers}.
This is not far from what people like Jim Simons (co-founder of the Simons Foundation) did.
Jim Simons is partially responsible for a lot of results in physics and astronomy (and mathematics and Autism research, and more), more than any astrophysicist,
and he did all that by running an extremely successful (and tax-avoidant) hedge fund in the US \cite{simons}.
Thus anyone working in astrophysics is someone who wants to \emph{do astrophysics}, not someone who wants to \emph{learn the answers}; there are way more efficient ways to learn the answers.\footnote{%
This argument assumes that astrophysicists are, by and large, rational.
That's a bad assumption, given the points about remuneration.
However, I think the point is still good.
Also, I must comment that this argument is unfortunately related to an absurdity known in the tech sector as ``effective altruism'' \cite{effectivealtruism}.
For this I apologize; I think that effective altruism is wrong because of its assumptions about predictability of impacts, effects, and needs over long time scales.
(It also assumes that the evil of rising disparity can be overcome with individual good works, but that's even \emph{more} off-topic.)
The argument presented here does not make assumptions about the predictability of medium-term or long-term futures.}
This all would be different, I believe, if astrophysics had a right edge (clinical value):
Then we would care about the detailed results, because, if they came out well, they would lead to new opportunities for humanity.

\paragraph{Stand back and let the AIs cook:}
One possible extreme policy proposal is that we \emph{fully embrace} the LLMs, and encourage them (or even give them enormous resources\footnote{%
One thing I have not addressed at all in this \documentname{} is the tremendous environmental impact of the training of LLMs, and their inferences (their execution as generative models working on prompts).
The numbers are staggering.
It is important to remember that the human practice of astrophysics, at least in its current form, is also very damaging to the environment \cite{low-carbon, imperative}.\label{note:environment}})
to conceive of, design, execute, write up, and submit (to the journals) novel astrophysics projects and papers.
Under this policy, the vast, vast majority of astrophysics would be done by machines, and we (as a community of humans) would participate primarily through reading, vetting, and discussing the novel astrophysical results produced by those machines.
Dear reader, you probably either think this policy is disastrously wrong, or else obviously right.
I want to take it seriously, not least because it is the literal, explicit proposal of one recent project \cite{denario}, and because that project has built a first version of the key tools necessary to implement this policy, end-to-end.

If we implement the full-acceptance policy, astrophysics changes dramatically, but it is consistent with some of our points of agreement (listed in \secref{sec:facts}), and it does deliver many of the key benefits of astrophysics (listed in \secref{sec:benefits}).
On the points of agreement: Astrophysics-by-LLM produces new knowledge; it doesn't use people (therefore it doesn't make people its ends); it contributes to the literature; it is efficient (time-efficient; it does use energy resources; see \noteref{note:environment}).
It is not clear that it will be rigorous and correct.
Right now it is hard to assess the correctness of LLM-generated content, or at least it is not easier to figure that out for LLM-generated content than any other kinds of refereeing.
At the rate at which LLMs can design and produce papers, we humans probably can't keep control of rigor and correctness.
Aside from that, astrophysics by LLM is indeed properly astrophysics, according to most of our points of agreement.

On the benefits: Astrophysics-by-LLM is almost as good as human astrophysics in terms of feeding humanity's love of the subject;
it delivers benefits to physics (provided that it is asked to do so);
it can be used for military development;
it can contribute to remote sensing;
it creates new knowledge (provided that the projects it creates are indeed novel).
That is, astrophysics-by-LLM delivers many of the benefits that we have identified and associated with the practice of astrophysics.

So is everything fine? No everything is \emph{not fine}.
Recall (from our points of agreement in \secref{sec:facts}) that the practice of astrophysics cannot be learned from \emph{reading about astrophysics}.... HOGG IS THIS ARGUMENT STRONG ENOUGH?
HOGG: IF IT IS, then this would be the death of astrophysics, the end of atrophysics at universities, and the eventual end of astrophysics education.
Astrophysics would no longer be \emph{by} humans, and then it would no longer be \emph{for} humans.

\paragraph{Ban and punish:}
The other possible extreme policy proposal is that we \emph{fully outlaw} the LLMs, and do not permit anyone in astrophysics to use them for anything other than the most menial of tasks (like summarizing documentation), and enforce negative consequences for people or projects that use them.
Under this policy, nothing fundamental about the practice of astrophysics would different from what it was in 2019, say, with the exception that web searches for code issues, and code development environments, might be a bit better (or worse, or at least different).
Looking outward, the ban-and-punish attitude is (I believe, based on anecdotal evidence) most ascendant in the traditional humanities (for example, English, or history), and least ascendant in the professional schools (for example, museum studies, or management).

The one difference in the practice of astrophysics (relative to the practice in 2019, say), if the astrophysics community fully implemented the ban-and-punish policy, would be that we would have to spend significant energy, resources, and attention on \emph{policing our communities}.
After all, as the the LLMs get better, their work gets harder and harder to detect (or to distinguish from human activity), so if we, as a community, wanted to implement ban-and-punish, we would put our ... HOGG WRITE WORDS.\footnote{HOGG Something about \textsl{Turnitin} being the back side of a cheating site.}

Ban-and-punish is an extreme policy, but it is consistent with with many of our points of agreement (listed in \secref{sec:facts}), and it doesn't interfere with most of the key benefits of astrophysics (listed in \secref{sec:benefits}):
... HOGG STUFF HERE.

Of course, HOGG ...

\paragraph{Is there a middle way?}
HOGG: I am not sure there is.
And even if there is, I am not sure I can find it.
And even if I can find it, I don't see how we, as a community, could agree to it (and even less police it).

\paragraphskip
PEOPLE: If you have read this far, you have read everything I have been able to coherently write, and lots I haven't.
Now I am just a mess.

HOGG:
Is there room to talk about Phil Marshall's point about humans ``providing meaning'' to work?

HOGG:
Is there room to talk about Natalie Hogg's point about LLMs being plagiarists?

HOGG:
How to work in a discussion of the post-structuralist idea that the meaning of a text is set by the reader, or its reception, and not by the author \cite{deadauthor}?
This idea is relevant here, because if the author doesn't matter to meaning, then LLMs can write ``just as well'' as humans, maybe?
Related also to PJM's point about meaning.
I guess PJM is not a post-structuralist?
Do we have to reject post-structuralism also if we reject the LLMs?

\bibliography{why}

\end{document}
